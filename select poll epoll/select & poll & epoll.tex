\documentclass[a4paper,fontset=mac]{ctexart}
\linespread{1.5}                         %行距
\usepackage[top=2cm,bottom=2cm,left=2.5cm,right=2.5cm]{geometry}
% \headsep=2cm
% \textwidth=16cm \textheight=24.2cm
\usepackage{fontspec,xltxtra,xunicode}
\usepackage[colorlinks,linkcolor=blue,anchorcolor=red,citecolor=green,urlcolor=blue]{hyperref}  
\usepackage{tabularx}
\usepackage{amsmath}                   % 数学符号与公式
\usepackage{amsfonts}                  % 数学符号与字体
\usepackage{graphics}
\usepackage{subfigure}
\usepackage{color}
\usepackage{fancyhdr}                  % 设置页眉页脚
\usepackage{fancyvrb}                  % 抄录环境
\usepackage{float}                     % 管理浮动体
\usepackage{geometry}                  % 定制页面格式
\usepackage{hyperref}                  % 为PDF文档创建超链接
\usepackage{lineno}                    % 生成行号
\usepackage{listings}                  % 插入程序源代码
\usepackage{multicol}                  % 多栏排版
\usepackage{rotating}                  % 旋转文字,图形,表格
\usepackage{subfigure}                 % 排版子图形
\usepackage{indentfirst}               % 首段缩进
\usepackage{booktabs}                  % 使用\multicolumn
\usepackage{multirow}                  % 使用\multirow
\usepackage{graphicx}  
\usepackage{xcolor}
\usepackage{cite}
\usepackage{listings}



\lstset{ %
	backgroundcolor=\color[RGB]{245,245,244},   % choose the background color; you must add \usepackage{color} or \usepackage{xcolor}
	basicstyle=\footnotesize,        % the size of the fonts that are used for the code
	breakatwhitespace=false,         % sets if automatic breaks should only happen at whitespace
	breaklines=true,                 % sets automatic line breaking
	captionpos=bl,                    % sets the caption-position to bottom
	commentstyle=\color[RGB]{0,96,96},    % comment style
	deletekeywords={...},            % if you want to delete keywords from the given language
	escapeinside={\%*}{*)},          % if you want to add LaTeX within your code
	extendedchars=true,              % lets you use non-ASCII characters; for 8-bits encodings only, does not work with UTF-8
	frame=none,                    % adds a frame around the code
	keepspaces=true,                 % keeps spaces in text, useful for keeping indentation of code (possibly needs columns=flexible)
	keywordstyle=\color[RGB]{40,40,255},       % keyword style
	%language=Python,                 % the language of the code
	morekeywords={*,...},            % if you want to add more keywords to the set
	numbers=left,                    % where to put the line-numbers; possible values are (none, left, right)
	numbersep=5pt,                   % how far the line-numbers are from the code
	numberstyle=\tiny\color{gray}, % the style that is used for the line-numbers
	rulecolor=\color{black},         % if not set, the frame-color may be changed on line-breaks within not-black text (e.g. comments (green here))
	showspaces=false,                % show spaces everywhere adding particular underscores; it overrides 'showstringspaces'
	showstringspaces=false,          % underline spaces within strings only
	showtabs=false,                  % show tabs within strings adding particular underscores
	stepnumber=1,                    % the step between two line-numbers. If it's 1, each line will be numbered
	stringstyle=\color[RGB]{128,0,0},     % string literal style
	tabsize=2,                       % sets default tabsize to 2 spaces
	language=c++,
	%title=myPython.py                   % show the filename of files included with \lstinputlisting; also try caption instead of title
}


\graphicspath{{sources/}}
\setlength{\parindent}{2em}       %设置缩进为两个大写M的宽度,大约为两个汉字的宽度
\ctexset{section = {format = \raggedright\Large\bfseries,}} %section 左对齐
\graphicspath{{sources/}}


%------------------------------设置字体大小------------------------%  
\newcommand{\cuhao}{\fontsize{42pt}{\baselineskip}\selectfont}     %初号  
\newcommand{\xiaocuhao}{\fontsize{36pt}{\baselineskip}\selectfont} %小初号  
\newcommand{\yihao}{\fontsize{28pt}{\baselineskip}\selectfont}      %一号  
\newcommand{\erhao}{\fontsize{21pt}{\baselineskip}\selectfont}      %二号  
\newcommand{\xiaoerhao}{\fontsize{18pt}{\baselineskip}\selectfont}  %小二号  
\newcommand{\sanhao}{\fontsize{15.75pt}{\baselineskip}\selectfont}  %三号  
\newcommand{\sihao}{\fontsize{14pt}{\baselineskip}\selectfont}       %四号  
\newcommand{\xiaosihao}{\fontsize{12pt}{\baselineskip}\selectfont}  %小四号  
\newcommand{\wuhao}{\fontsize{10.5pt}{\baselineskip}\selectfont}    %五号  
\newcommand{\xiaowuhao}{\fontsize{9pt}{\baselineskip}\selectfont}   %小五号  
\newcommand{\liuhao}{\fontsize{7.875pt}{\baselineskip}\selectfont}  %六号  
\newcommand{\qihao}{\fontsize{5.25pt}{\baselineskip}\selectfont}    %七号


\setCJKmainfont[BoldFont={STHeiti}, ItalicFont={STKaiti}]{STSong}
%\setCJKsansfont{SimHei}
\setCJKsansfont{STHeiti}
%\setCJKmonofont{FangSong}
\setCJKmonofont{STFangsong}

\setmainfont{Times New Roman}   %西文默认衬线字体(serif)
\setsansfont{Arial}   %西文默认无衬线字体(sans serif)
\setmonofont{Courier New}           %西文默认的等宽字体




\title{select、poll,epoll笔记}
\author{sharwen}
\date{}
%==================================================
%%%%%%%%%%%%%%%%%%%%%%%%%%%%%%%%%%%%%%%%%%%%%%%%%%%
% 正文
%==================================================
\begin{document}
	% \pagenumbering{Roman}          %页码为大写罗马数字
	% \pagenumbering{arabic}         %页码为阿拉伯数字
%	{\tiny \thispagestyle{empty}   %本页不显示页码
%	\begin{figure}[h]
%		\centering
%		\includegraphics[width=5in]{logo1}
%		\includegraphics[width=2in]{logo2}
%	\end{figure}
%	
%	\vspace{42pt}
%	\begin{table}[h]
%		\centering
%		\linespread{2}   
%		\sanhao
%		\begin{tabular}{lcl}% 通过添加 | 来表示是否需要绘制竖线
%		%		\hline  % 在表格最上方绘制横线
%		\textbf{题\, \; \; \; 目}&\textbf{:}&\textbf{论文综述题目}\\
%		\textbf{姓名学号}&\textbf{:}&\textbf{罗胜文/21821167}\\
%		\textbf{邮\, \; \; \; 箱}&\textbf{:}&\textbf{sharwen568@gmail.com}\\
%		\textbf{电\, \; \; \; 话}&\textbf{:}&\textbf{18883287841}\\
%		\textbf{老\, \; \; \; 师}&\textbf{:}&\textbf{。。。}\\
%		\textbf{专\, \; \; \; 业}&\textbf{:}&\textbf{计算机科学与技术}\\
%		\end{tabular}
%	
%		\vspace{42pt}
%		\centering
%		\textbf{2019年1月9日}
%	\end{table}
%	
%
%	\newpage
%	\setcounter{page}{1}}
	\maketitle
	\tableofcontents
%	\begin{abstract}
%		abstract goes here.
%		
%		\centering%使得关键字居中
%		\textbf{关键字:}摘要、\LaTeX、中文
%	\end{abstract}
	\newpage%另起一页
	
	\section{select}
	select是Linux内核自身支持的一个网络编程,支持同时多fd监听的网络编程接口。
	
	\subsection{select接口}
	\begin{lstlisting}
	int select(int nfds, fd_set *readfds, fd_set *writefds,	fd_set *exceptfds, struct timeval *timeout);
	void FD_CLR(int fd, fd_set *set);
	int  FD_ISSET(int fd, fd_set *set);
	void FD_SET(int fd, fd_set *set);
	void FD_ZERO(fd_set *set);
	
	/* fd_set for select and pselect.  */
	#define __FD_SETSIZE 1024
	
	/* The fd_set member is required to be an array of longs.  */
	typedef long int __fd_mask;
	
	/* It's easier to assume 8-bit bytes than to get CHAR_BIT.  */
	#define __NFDBITS (8 * (int)sizeof(__fd_mask)) // 每 long int 的位数 32
	#define __FD_ELT(d) ((d) / __NFDBITS)
	#define __FD_MASK(d) ((__fd_mask)1 << ((d) % __NFDBITS)) // 取模进行左移
	
	typedef struct {
	__fd_mask __fds_bits[__FD_SETSIZE / __NFDBITS];
	#define __FDS_BITS(set) ((set)->__fds_bits)
	} fd_set;
	
	#define __FD_SET(d, set) ((void)(__FDS_BITS(set)[__FD_ELT(d)] |= __FD_MASK(d)))
	#define __FD_CLR(d, set) ((void)(__FDS_BITS(set)[__FD_ELT(d)] &= ~__FD_MASK(d)))
	#define __FD_ISSET(d, set) ((__FDS_BITS(set)[__FD_ELT(d)] & __FD_MASK(d)) != 0)
	\end{lstlisting}
	从代码易知道,select支持的最大fd数量为{\color{red}1024}个,fd\_set使用位运算来标识 0<=fd < 1024,如果超过这个范围,则会引发未知错误如程序崩溃等。这个\_\_FD\_SETSIZE也可以自己手动修改,以便支持更多的fd数量。虽然一般单个程序fd的标号不会超过1024,但也存在一个程序处理超过这么多的fd,因此这个隐患还是很明显的。
	
	
	\subsection{为什么select慢}
	在第一次所有监听都没有事件时,调用 select 都需要把进程挂到所有监听的文件描述符一次,并切调用select会将所有的fd从用户空间拷贝到内核空间。
	
	有事件到来时,不知道是哪些文件描述符有数据可以读写,需要把所有的文件描述符都轮询一遍才能知道。
	
	通知事件到来给用户进程,需要把整个 bitmap 拷到用户空间,让用户空间去查询。
	
	
	
	
	
	
	

	
	\newpage%另起一页
	\begin{thebibliography}{1}
		\bibitem{}
		STL源码剖析,侯捷等
	\end{thebibliography}
	
% 插入图片并排,且图片编号不属于同一个大编号,图片编号数字不同
%	\begin{figure}
%		\begin{minipage}[t]{0.5\linewidth}
%			\centering
%			\includegraphics[width=2.2in]{logo1}
%			\caption{fig1}
%			\label{logo1}
%		\end{minipage}%
%		\begin{minipage}[t]{0.5\linewidth}
%			\centering
%			\includegraphics[width=2.2in]{logo2}
%			\caption{fig2}
%			\label{logo2}
%		\end{minipage}
%	\end{figure}



% 插入图片并排,图片属于同一个编号,但是每一个图片有自己的子编号a,b,c
%\begin{figure}
%	\centering
%	\subfigure[logo1]{
%		\label{logo1} %% label for first subfigure
%		\includegraphics[width=1.0in]{logo1}}
%	\hspace{1in}
%	\subfigure[logo2]{
%		\label{logo2} %% label for second subfigure
%		\includegraphics[width=1.5in]{logo2}}
%	\caption{Two Subfigures}
%	\label{logo} %% label for entire figure
%\end{figure}


	\end{document}
%==================================================
%%%%%%%%%%%%%%%%%%%%%%%%%%%%%%%%%%%%%%%%%%%%%%%%%%%